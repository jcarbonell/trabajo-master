\chapter{Material y métodos}


\section{Obtención de genes candidatos}

La metodología propuesta en el presente trabajo trata de describir cómo priorizar un conjunto de genes candidatos obtenidos a partir de un típico estudio caso/control. El contexto de aplicación serán las enfermedades hereditarias y se va a plantear una metodología compatible tanto con un estudio de tipo familiar, donde se dispone de una o más familias distintas con el mismo síndrome (aunque no necesariamente con la misma mutación), cómo con un estudio de tipo más general, donde los individuos disponibles no muestran parentesco alguno. 

\medskip
Cuando afrontamos un estudio de tipo familiar, los individuos seleccionados, tanto casos cómo controles, se distribuyen a lo largo de las distintas familias. Si se ha realizado un diseño experimental razonable, cada nucleo familiar, dispondrá tanto de sujetos caso, cómo de sujetos control propios. La ventaja proporcionada por las estructuras familiares es que a priori se espera que todos los individuos enfermos pertenecientes a la misma familia compartan la misma alteración genómica, lo cual sería mucho más difícil de afirmar en el caso de individuos independientes. Esta situación permite reducir drásticamente el número de candidatos finales con sólo realizar una simple intersección entre las mutaciones de los individuos enfermos. 

\medskip
Por otro lado, los familiares sanos incluidos en el estudio dispondrán de una capacidad de filtrado mucho mayor, ya que al compartir con sus familiares enfermos una porción de genoma mayor de lo esperable con un control externo, será posible eliminar una cantidad mayor de mutaciones no relacionadas con la enfermedad. Además, dicha potencia aumentará con el grado de parentesco, se calcula que un hermano sano (el cual compartirá aproximadamente el 50\% de su genoma con su hermano enfermo) puede tener una capacidad de filtrado equivalente a cientos de controles externos.

\medskip 
Finalmente, se dispone de una lista de genes candidatos por cada grupo de individuos independientes. Si el estudio es familiar, dispondremos de una lista de candidatos para cada familia, mientras que en el caso general, cada individuo aportará su lista de genes.

%%% INDEPENDENCIA!!!

\section{Metodología general}

\subsection{Esquema general del método}

La metodología propuesta puntúa a cada gen candidato en función de su relevancia global sobre el total de familias o individuos estudiados. Básicamente, se trata de computar la frecuencia de aparición de cada candidato sobre las listas de genes independientes y complementar este valor con un término proporcional a la cantidad de interacciones descritas entre el candidato y dichos genes. 

\medskip
El término de interacción trata de incrementar el peso de un gen sobre una familia cuando este no ha sido explicitamente seleccionado cómo un candidato, de forma que, en caso de existir evidencias claras de interacción con uno o varios de sus genes, se puede afirmar que el gen también resulta importante para la familia .

%\medskip
%Una forma sencilla de abordar el problema cuando esperamos que todos los individuos del estudio tengan mutado el mismo gen, sería mediante la intersección directa de las listas de %candidatos proporcionadas por cada familia. En un escenario sencillo, esta sería probablemente la metodología más adecuada, ya que, salvo por un error de secuenciación o %procesamiento posterior, el gen de la enfermedad estará presente en todas las listas de candidatos obtenidas. Además, al aumentar el número de familias en el estudio se estará %aumentando de forma exponencial la fiabilidad del resultado, ya que, se reduce la probabilidad de encontrar un gen que por azar esté presente también en todas las familias. Sin %embargo, cuando disponemos de familias o individuos heterogéneos, con el mismo síndrome, pero con genes mutados diferentes, la metodología de intersección directa no sería valida. %El problema se agrava al considerar que los errores del protocolo podrían provocar que el gen de la enfermedad no fuera correctamente secuenciado y que por tanto, no apareciera en %la lista de candidatos iniciales. En este caso, es necesario aplicar metodologías más sofisticadas que permitan detectar cuando un gen, aun no habiendo sido seleccionado %directamente por una familia, presenta interacciones descritas con los genes que sí han sido seleccionados.

\medskip
Aproximadamente un 15\% de enfermedades mendelianas disponen de más de un gen distinto con capacidad para producir la misma enfermedad, este porcentaje aumentaría mucho en el caso de las enfermedades complejas, donde un mismo gen mutado combinado con otros genes distintos, puede jugar un rol importante en varios síndromes. 

\medskip
La biología de sistemas demuestra que los genes actúan de forma colaborativa para llevar a cabo las distintas tareas esenciales para la célula. Los genes se comunican en procesos biológicos donde el grado de coordinación entre las distintas moléculas participantes es muy alto. Esto implica que si un síndrome está causado por el mal funcionamiento de una ruta metabólica, es probable que al alterar alguno de sus genes importantes, se acabe produciendo el mismo resultado para el individuo. 

*** imagen pathway

\medskip
Hay que señalar que, a menudo, las células disponen de mecanismos naturales de compensación que les permiten sobrevivir pese al mal funcionamiento de alguno de sus genes importantes. Desgraciadamente, estos mecanismos no siempre consiguen evitar la aparición de anomalias graves que posteriormente provocarán la aparición de una enfermedad.

\medskip
El proceso de priorización comienza con el reclutamiento de todas las listas de genes candidatos aportadas por cada familia.

\begin{equation}
  C = [c_1,c_2,...,c_i,..,c_n] 
\end{equation}

\medskip
donde $C$ se corresponde con el vector de listas de candidatos, $n$ con el número de familias (o grupos independientes) del estudio y $c_i$ con la lista de candidatos aportada por la familia $iesima$.
\\

A continuación, se procede a construir el set global de candidatos $G$ formado a partir de la unión de todas las listas independientes

\begin{equation}
G = unico( \cup \; \forall c_i, i \exists C )
\end{equation}

\medskip
Seguidamente, se procede a calcular el estadístico de priorización para cada gen contenido en el set global de candidatos:

\begin{equation}
\rho_i = \Phi(g_i), \forall g_i, i \exists G 
\end{equation}

\medskip
Por último, se ordena la lista global de candidatos en función del estadístico de priorización computado.

\begin{equation}
R = orden(\rho)
\end{equation}

\medskip
De esta forma, los genes que estén en lo alto de la lista serán los mejor priorizados, y por tanto, los primeros a validar.

\medskip
El cómputo del estadístico de priorización para un determinado gen candidato, se realiza básicamente a partir de la suma del peso que tiene el gen en cada una de las familias del estudio. 

\begin{equation}
\rho_i =  \sum_{j=1}^{n}  \Phi(g_i,c_j)
\end{equation}

\medskip
donde $\Phi(g_i,c_j)$ se corresponde con el peso del gen $i$ sobre la familia $j$

\medskip
Cuanto mayor sea el peso del gen en las distintas familias, o mayor sea el número de familias en las que el gen está presente, mejor será su ponderación. Para calcular la relación entre un gen candidato y una familia de estudio, se realiza una evaluación en dos partes: en primer lugar se proporciona un peso inicial en función de si el gen está presente en la lista de candidatos aportada por la familia y en segundo lugar, se añade un segundo término proporcional a la cantidad de interacciones descritas entre el gen candidato y los genes seleccionados por la familia. Concretamente:

\begin{equation}
\rho_{i,x} = \sum_{j=1}^{n} \alpha_j * \gamma_{ij} + \delta(i,j) * (1-\gamma_{ij})
\end{equation}

\medskip
donde \textit{n} se corresponde con el número de familias, $\alpha_j$ con el peso inicial asociado a la familia \textit{j}, $\gamma_{ij}$ con un factor con valores 0 o 1 en función de si el gen \textit{i} está seleccionado por la familia \textit{j} y $\delta(i,j)$ con la función que estima el grado de interacción entre el gen \textit{i} y el vector de genes seleccionado por la familia \textit{j}.


\subsection{Estimación del grado de interacción entre genes}

La parte más compleja del proceso consiste en cómo estimar el grado de interacción entre un gen candidato y el conjunto de genes aportados por una familia. Para ello, es necesario disponer de una base de datos que recoja el total de interacciones descritas entre los genes. En este caso, se hará uso de diferentes interactomas (o redes de interacción génica). Se trata de bancos de datos que recogen todas las interacciones descritas entre cada par de genes, a partir de los cuales es posible reconstruir fácilmente la red de interacción global de todas las proteínas o genes. 

\medskip
La red de interacción, donde los nodos son los genes y las aristas sus interacciones, se define cómo:


\begin{equation}
I = (V,E)
\end{equation}

donde $I$ se corresponde con el grafo general, $V$ con el conjunto total de genes, y $E$ con sus interacciones descritas.

\medskip
La red permite recuperar los vecinos directos de un determinado gen, pero también reconstruir totalmente los caminos o secuencias de genes que podríamos emplear para llegar desde un nodo (o gen) de la red a otro:

\begin{equation}
 P_{i,j} = [P_{i,j,0},P_{i,j,1},...,P_{i,j,t}]
\end{equation}
\\

donde $P_{i,j}$ se corresponde con el conjunto total de $t$ caminos entre el nodo $V_i$ y el nodo $V_j$.
\\

Asimismo, cada camino se compone del conjunto de interacciones entre ambos genes.

\begin{equation}
P_{i,j,\alpha} = [E_{i,0},E_{0,1},E_{1,2},...,E_{k-1,k},E_{k,j}], V_i \in V, V_j \in V
\end{equation}

done $P_{i,j,\alpha}$ se corresponde con un posible camino del nodo $V_i$ al nodo $V_j$ compuesto por las interacciones entre ambos nodos y los $k$ intermediarios necesarios.

\medskip
El conjunto total de caminos existentes entre cada par de genes de la red nos va a permitir calcular una medida de distancia que va a ser directamente empleada para estimar el grado de interacción entre ellos. 

\begin{equation}
d_{i,j} = f(P_{i,j})
\end{equation}

\medskip
Intuitivamente, una distancia pequeña o un número grande de caminos posibles describirá una interacción fuerte entre dos genes, mientras que un número elevado de intermediarios o un número pequeño de caminos posibles describirán una interacción pobre entre los mismos. 

\medskip
En la práctica, se emplearán varios interactomas distintos, encargados de recoger interacciones de diferente naturaleza. Concretamente, para el presente trabajo han sido empleados los siguientes interactomas:

\begin{itemize}
\item \emph{Binding}: interacción física entre las proteínas generadas por dos genes
\item \emph{Ptmod}: modificaciones post-transcripcionales
\item \emph{Functional}: funciones comunes
\item \emph{Regulation}: relaciones de tipo regulador-regulado
\item \emph{Text-mining}: relaciones entre genes obtenidas a partir de artículos y técnicas de minería de datos
\end{itemize}

\medskip
Es importante señalar que el cómputo del grado de interacción descrito con anterioridad se realiza de forma independiente para cada interactoma, por lo que se obtendrán tantos rankings cómo interactomas hayan sido empleados en el estudio. Así pues, una de las tareas importantes dentro de la metodología propuesta consiste en cómo unir o ponderar los resultados obtenidos con cada interactoma para obtener un resultado global.

\begin{equation}
\rho_{i} = f([\rho_{i,0},\rho_{i,1},\rho_{i,m}])
\end{equation}

donde $\rho_{i}$ se corresponde con el estadístico de priorización global obtenido para el gen $i$, a partir del valor indivual obtenido en cada uno de los $m$ interactomas empleados.

\medskip
Conocer a priori cual es el interactoma que mejor recoge las relaciones existentes entre los genes implicados en una enfermedad es una tarea complicada, ya que depende de la naturaleza misma del síndrome. Algunas enfermedades mendelianas pueden ser descritas de forma casi completa mediante la alteración de un conjunto de interacciones físicas entre sus proteínas, sin embargo, otras enfermedades, principalmente complejas, serían mejor descritas por un interactoma de regulación. 

\medskip
Por otro lado, es importante indicar que no todos los interactomas describen en realidad interacciones directas entre los genes. En la práctica, se emplea el mismo tipo de representación para describir cualquier tipo de relación entre dos nodos, cómo por ejemplo, una evidencia de coexpresión entre dos genes. Esta representación permite evaluar de forma sencilla las relaciones indirectas entre dos genes. Si por ejemplo el gen \emph{A} y el gen \emph{B} se expresan simultáneamente en algún tejido y el gen \emph{B} y el gen \emph{C} se expresan simultáneamente en algún momento del desarrollo, es facil inferir que \emph{A} y \emph{C} tienen una estrecha relación y que podrían incluso coexpresar bajo condiciones muy determinadas. De esta forma, podríamos concluir que \emph{A} y \emph{C} están a una distancia mucho menor de la que, en promedio, encontraríamos entre dos genes escogidos de forma aleatoria.
 
\medskip
También es importante resaltar, que la gran mayoría de interacciones existentes entre dos proteínas, son todavía desconocidas. Esto significa que los interactomas disponibles muestran una falta de completitud bastante grande, que en la práctica podría suponer un handicap considerable a la hora de realizar la priorización. 
 
\subsection{Búsqueda de vecindarios compartidos entre familias}

La forma en que se desarrolla todo el proceso de secuenciación y su posterior análisis estadístico provoca que, en general, se disponga de un número de falsos positivos muy elevado en relación al número de positivos esperados. Tanto si se trabaja con familias, cómo con individuos independientes, es conveniente aumentar el tamaño muestral ya que, debido a la metodología de intersección y filtrado empleada, este tiene un impacto directo sobre la talla del conjunto de genes candidatos a priorizar y por tanto, sobre la precisión final del resultado obtenido.

\medskip
Si se realiza un estudio familiar con una enfermedad mendeliana, únicamente se espera encontrar un gen responsable por familia, ya que todos los individuos enfermos deberían coincidir en su mutación nociva. El resto de genes seleccionados cómo candidatos se corresponden con falsos positivos obtenidos principalmente a causa de limitaciones derivadas del tamaño muestral máximo que una familia normal puede ofrecer. Hay que tener en cuenta que, si disponemos de un diseño experimental razonable, el número de genes totales aportados por cada familia estará entre 20 y 200 genes, lo cual significa que, en el mejor de lo casos, estaremos introduciendo aproximadamente un 95\% de falsos positivos. 

\medskip
En la metodología propuesta, la intersección realizada entre familias debería descartar de forma clara la mayor parte de genes aleatorios propios de cada familia, de forma equivalente a cómo se reduce el ruido al promediar varias adquisiciones de la misma señal. 

\medskip
Tanto si se trabaja con una enfermedad mendeliana, o sobre un síndrome complejo, es probable encontrar, en el conjunto total de individuos, más de un gen distinto que de forma directa o indirecta regule los procesos biológicos alterados en el síndrome de estudio. Si las interacciones existentes entre los genes causantes estuvieran correctamente descritas en un interactoma, significa que se acabaría observando un cluster o vecindario en la red con una densidad de genes candidatos por encima de lo normal. 

\medskip
El hecho de que la mayoría de genes aportados por las familias sean de carácter aleatorio permite sugerir que el grado de interacción medio para una red debería ser equivalente al obtenido en promedio para un grupo de genes escogidos de forma aleatoria. Eso significa que el vecindario que contiene al grupo de genes causantes tendrá un valor de priorización medio significativamente superior a la media y que por tanto debería ser identificable. Sin embargo, en la práctica resulta más complicado ya que si el proceso de selección de los candidatos iniciales no consigue acotar de forma considerable el conjunto de genes a evaluar, surgirán otros vecindarios altamente conectados simplemente por azar. 

\medskip
Otro aspecto a tener en cuenta es que la región de la red donde figuran los genes a identificar podría no estar descrita de forma completa en el interactoma empleado, lo que provocaría un descenso en la conectividad media del cluster y por tanto una subestimación del estadístico de priorización para el grupo de genes causante de la enfermedad.


\section{Estimación de parámetros}
	
	\subsection{Medidas de distancia}
	
	El estadístico de priorización incorpora un término matemático que recoge el grado de interacción entre el gen y cada una de las familias incluidas en el estudio. El grado de interacción reposa directamente sobre la medida de distancia calculada entre el gen de estudio y cada uno de los candidatos familiares. En la práctica, estimar la distancia entre dos genes dentro de una red de interacción no es algo trivial, ya que, debido a que los interactomas son redes altamente conectadas, lo normal es disponer de más de un camino distinto para llegar de un punto a otro de la red. 
	
\medskip
La función de distancia toma cómo entrada al conjunto formado por los N caminos disponibles entre ambos nodos (ecuación 2.10). El caso es especialmente complicado cuando se dispone de una gran cantidad de caminos, con longitudes muy distintas. Para este trabajo, se han implementado y validado 3 medidas de distancia diferentes que recogen varios enfoques distintos a la hora de cuantificar la proximidad entre dos nodos.

\medskip
Las medidas empleadas son las siguientes:

\subsubsection{a) Shortest path (camino más corto)}

La distancia entre dos nodos viene definida por la longitud del camino más corto entre ellos, también conocido como distancia geodésica. 

\begin{equation}
SP_{i,j} = min(L(P_{i,j}))
\end{equation}
\\

donde $L$ se corresponde con la función de longitud computada sobre el conjunto de caminos $P_{i,j}$ y $SP_{i,j}$ su valor mínimo.

\medskip
Se trata de una medida que simplifica enormemente el cómputo de distancias, pero que deja de lado algunas características propias de la topología de la red, cómo el número de caminos existentes entre dos nodos.

\subsubsection{b) Intermediación}

Mide la frecuencia en la que el gen $i$ aparece en el total de caminos cortos que parten del gen $j$

\begin{equation}
IM_{i,j} = \frac{L(i \in SP_j)}{L(SP_j)} 
\end{equation}

donde $IM_{i,j}$ se corresponde con la frecuencia de intermediación de $i$ sobre $j$ y $SP_j$ el conjunto total de caminos cortos tomando como origen al gen $j$.
\\

En este caso, se considera que cuando un gen $i$ aparece de forma sistemática en los caminos que parten de $j$ hacia el resto de genes, significa que ambos interaccionan en un gran número de procesos celulares y que por tanto, están muy próximos, aunque el camino más corto que los une sea largo.

\subsubsection{c) Random walk}

La medidad de distancia viene determinada por la probabilidad de llegar al gen $j$, partiendo desde el gen $i$, cuando el \emph{viajero} elige caminos aleatorios.

\begin{equation}
RW_{i,j} = P(j|i)*P(j)
\end{equation}

Se trata de la adaptación del conocido algoritmo de \emph{random walk} para el trabajo con redes. Se trata de un método computacionalmente costoso, pero que tiene en cuenta la estructura total de la red de interacción.

Random walk (parámetros por defecto) ****


\subsection{Integración de los estadísticos computados por interactoma}

La comunicación entre genes puede producirse a diferentes niveles. Cada tipo de interacción se representa mediante el mismo modelo de red, pero en interactomas separados. Se trata de un sistema de información que puede ser enriquecido y actualizado de forma periódica, tanto con nuevas interacciones descritas en la literatura reciente, como a partir de interactomas nuevos, que describen otro tipo de relaciones no empleadas hasta ese momento. El hecho de almacenar tantos tipos de interacción como sea posible, permite disponer de un criterio biológico más amplio y preciso a la hora de evaluar la relación entre dos genes.
	
	\medskip
Después del proceso de priorización se dispone para cada gen de tantos valores como interactomas hayan sido incluidos en el estudio. De esta forma, el estadístico de priorización del gen $i$ para un interactoma $x$ se define como:


\begin{equation}
\rho_{i,x} = \sum_{j=1}^{n} \alpha_j * \gamma_{ij} + \delta(i,j,x)*(1-\gamma_{ij})
\end{equation}
\\

donde $\rho_{i,x}$ se corresponde con el valor de priorización para el interactoma $x$ y $\delta(i,j,x)$ como el grado de interacción entre el gen $i$ y la familia $j$ en ese mismo interactoma.
\\

 \medskip
 A su vez, el término de interacción $\delta$ se define como:
 
\begin{equation}
 \delta(i,j,x) = f([d(i, G_{j1},x), d(i,G_{j2},x),...,d(i,G_{jk},x)])
\end{equation}

   
  donde \textit{d} es la función de distancia entre el gen \textit{i} y un gen de la familia \textit{j} en el interactoma $x$ y \textit{f} la función que integra todas las medidas de distancia obtenidas y proporciona finalmente el valor de interacción entre el gen candidato \textit{i} y la familia \textit{j}. En este caso, se ha definido el valor de interacción gen/familia como la media de aquellas medidas de distancia con un valor superior al cuantil 0.95 de la distribución, es decir, la interacción entre un candidato $i$ y una familia $j$ se computa únicamente a partir de los genes con mayor proximidad.

\medskip
Por último, se computa el valor de priorización global tal que:

\begin{equation}
\rho_{i} = f([\rho_{i,1},\rho_{i,2},...,\rho_{i,m}])
\end{equation}
\\

donde $\rho_{i}$ se corresponde con el valor global de priorización para el gen $i$ obtenido a partir de un función $f$ que integra los valores de priorización de los $m$ interactomas.
\\

Desgraciadamente, en la práctica los genes no disponen de información para todos los interactomas disponibles, ya que, en general todos muestran en mayor o menor medida signos evidentes de falta de completitud. Debido a esto, es necesario determinar cuando el valor de priorización obtenido a partir de un interactoma aporta información, o por el contrario, provoca una subestimación del peso. Para el presente trabajo, se ha considerado que un interactoma no debe ser empleado cuando su valor de priorización es igual a 0, es decir, cuando no dispone de interacciones descritas para el gen. 

\medskip
Asimismo, se han empleado dos funciones integradoras distintas: la media y el máximo de los valores de priorización que no son 0. Cabe destacar que en el mejor de los casos, se dispondrá de tantos valores como interactomas, lo que puede impedir la aplicación de otras funciones integradoras de mayor complejidad.

\subsection{Otras ponderaciones complementarias al método}
	
La metodología propuesta comienza en el instante posterior a la selección inicial de candidatos por parte de cada familia. Hasta el momento, cada uno de los genes seleccionados por una familia muestra a priori la misma probabilidad de causar la enfermedad. Sin embargo, hay determinadas estrategias que pueden enriquecer o completar la metodología planteada estableciendo unas probabilidades a priori diferentes para cada gen. Estas estrategias pueden trabajar a partir de los propios datos del estudio, o con información conocida almacenada en bases de datos públicas, la cual permite estimar la importancia de cada gen dentro del contexto global de la célula y por tanto ponderar de forma positiva a aquellos genes que por su rol podrían acarrear consecuencias mucho peores a la célula en caso de mal funcionamiento. Estas medidas pueden llegar a corregir el valor de priorización obtenido en presencia de errores de secuenciación o por la falta de información en los interactomas.

\medskip
A continuación, se describen algunas estrategias posibles.

\subsubsection{Evaluación de las mutaciones del gen}
   
La reglas biológicas que rigen el proceso de traducción de un ARN mensajero en una proteína totalmente funcional, describen como una única mutación puede ser capaz de inutilizar o provocar el mal funcionamiento de un gen y como consecuencia, una cascada de anomalías que derive en una enfermedad. No obstante, esto no debería obviar el hecho de que aquellos genes que acumulen un mayor número de mutaciones nocivas, deberían tener una probabilidad mayor de contener a la mutación causante de la enfermedad, por lo que deberían ser a priori mejor ponderados.

\medskip
Por otro lado, se conoce que determinadas mutaciones en zonas codificantes, aun habiendo provocado un cambio de aminoacido en la secuencia de la proteína, en realidad no producen cambios significativos en su conformación y por tanto en su funcionamiento. En ese sentido, existen en la actualidad algunas herramientas informáticas disponibles, como SIFT \cite{sift} o Polyphen \cite{polyphen} que evalúan algunas características esenciales de la secuencia de aminoacidos mutada, y proporcionan un estadístico que describe el grado cambio en la proteína.

\medskip
Otra de las formas de evaluar el potencial efecto de una mutación consiste en determinar si esta ha sido descrita anteriormente en población sana no incluida en el estudio, ya que de ser así, podría no reunir las condiciones necesarias para producir la enfermedad. Para tal efecto, es posible consultar si la mutación ha sido recogida por dbSNP \cite{dbsnp}, lo cual probaría a priori su inocuidad, o consultar si ha sido descrita en el proyecto de los 1000 genomas \cite{1000genomes}, y en caso de ser así, con qué frecuencia alélica. Este dato resulta de gran utilidad, ya que, en general, las enfermedades complejas surgen a raíz de una combinación de mutaciones, que de forma individual sí pueden estar presentes en población sana.
  

  \subsubsection{Rol general del gen en la red de interacción}

El estadístico de priorización obtenido a partir de los interactomas depende totalmente del set de candidatos escogidos, de tal forma que un mismo gen, podría tener valores de ponderación muy diferentes, en función de los genes que le acompañen. Sin embargo, existen algunas medidas generales interesantes acerca del gen que pueden ser computadas de forma determinista a partir de un interactoma. Estas medidas permiten evaluar la importancia del gen en la red en función de parámetros como el número de conexiones. Una de las medidas que mejor describe el rol del gen dentro de la red de interacción lo constituye el concepto de centralidad, el cual trata precisamente de determinar la importancia relativa de un nodo en el contexto global del interactoma. 

\medskip
En la práctica, la centralidad puede ser computada de muchas maneras. Por ejemplo, se puede hacer uso de los siguientes indicadores:

\bigskip
\textit{a) Grado\\}

Número de conexiones existentes para el nodo. Es la medida más simple para describir la centralidad. A mayor número de conexiones, se le atribuye mayor importancia.

\begin{equation}
Grado_i = L([E_{i,1},E_{i,2},...,E_{i,k}])
\end{equation}

donde $L$ es la función de longitud sobre el vector de $k$ interacciones que contienen al gen $i$

\bigskip
\textit{b) Cercanía}\\

Suma (u en ocasiones media) de las distancias existentes entre un nodo y todos aquellos nodos accesibles. Se trata de una medida más compleja donde, a valores más pequeños, mayor cercanía y por tanto, mayor importancia.

\begin{equation}
Cercania_i = \sum_{j=1}^{k} d(c_{i,j}) / k
\end{equation}

donde $Cercania_i$ se corresponde con la media de la distancia de todos los $k$ caminos entre el gen $i$ y cualquier otro nodo $j$

\bigskip
\textit{c) Intermediación}\\

Frecuencia con la que un nodo aparece en el camino más corto entre cada par de nodos de la red. A mayor frecuencia, mayor importancia. Se trata del mismo concepto empleado como medida de distancia, pero computado de forma global para un gen y todos los nodos del interactoma.

\medskip
Otro de los enfoque actuales más interesantes para estimar la importancia de un nodo en la red lo constituyen los estadísticos empleados por los motores de búsqueda de internet para determinar la importancia de una \textit{web}.  El caso más popular lo representa el algoritmo PageRank de Google, el cual, debido a su generalidad, es directamente aplicable para estimar la importancia de un gen dentro de un interactoma. 

  
  \subsubsection{Información funcional}

Actualmente, se dispone de gran cantidad de información biológica relativa a los genes y los procesos biológicos en los que intervienen. Repositorios como Gene Ontology \cite{go}, o KEGG \cite{kegg} ofrecen información estructurada en forma de ontologías acerca de rutas metabólicas, rutas de señalización y otros procesos biológicos descritos en la literatura. Esta información puede ser de gran utilidad, ya que si el investigador responsable del estudio conoce a priori aquellas funciones biológicas en las que el gen de la enfermedad debería estar implicado, se podría llevar a cabo un filtrado drástico de forma directa.  El problema de esta metodología es que no se puede sistematizar con facilidad, ya que el proceso requiere de la intervención del investigador para definir las funciones clave, que en ocasiones, estarán descritas de forma diferente según el repositorio de consulta.


\subsection{Diagrama general del método}


\bigskip

\begin{flushleft}
\noindent Entrada: \\
\line(1,0){350}
\end{flushleft}

\noindent \hspace*{1cm} G = [g1,g2,...gj] $\rightarrow$ lista de genes candidatos por familia \\
\hspace*{1cm} I = lista de interactomas \\ 
\hspace*{1cm} PC = priorizaciones complementarias \\

\begin{flushleft}
\noindent Algoritmo: \\
\line(1,0){350}
\end{flushleft}

\noindent \# Construcción del set global de genes candidatos
\\
$C$ = union($G$) \\
\\
\# Cómputo de estadísticos de priorización \\
Para todo gen $i$ contenido en el set de candidatos $C$ \\
\\
\hspace*{1cm} Para todo interactoma $x$ \\
	\\
	\hspace*{2cm} $p_{i,x} = 0$ \\
	\\
	\hspace*{2cm} Para toda familia $j$ \\
		\\
		\hspace*{3cm} $p_{i,j,x}=$ computo\_estadístico ( $i$, $j$,$x$ ) \\
		\hspace*{3cm} $p_{i,x} = p_{i,x} + p_{i,j,x}$ \\
		\\
	\hspace*{2cm} fin \\
	\\
	\hspace*{2cm} \# Cómputo del estadístico global \\
	\hspace*{2cm} $p_i=$ computo\_estadístico\_global ( $p_{i,x}$ ) \\
	\\
	\hspace*{2cm} \# Repriorización con estadísticos complementarios \\
	\hspace*{2cm} Para todo estadístico complementario $pc$ contenido en PC
		\\
		\hspace*{3cm} $p_i$ = $p_i * pc(i)$
		\\
	\hspace*{2cm} fin \\
	\\
\hspace*{1cm} fin \\
fin

\bigskip




\section{Validación}

Después de plantear en detalle la metodología de priorización, es necesario realizar una serie de experimentos que permitan evaluar el procedimiento de forma global y la influencia de cada parámetro característico sobre el resultado. Dicha validación se ha realizado en dos partes, en primer lugar se han empleado simulaciones para evaluar de forma exhaustiva cada parámetro del estudio, y por último, la metodología propuesta se ha aplicado en un caso real donde se conoce el gen causante de la enfermedad.

	\subsection{Simulaciones}
	
Las simulaciones nos permiten evaluar de forma exhaustiva el rendimiento de los parámetros del método, los cuales, a partir de datos reales costarían mucho de calibrar. Con el fin de simular de forma realista un caso de estudio típico, los genes de enfermedad seleccionados para cada simulación han sido extraidos de síndromes reales. Concretamente, se han extraído a partir del repositorio OMIM, mediante el cual se preparó una lista de enfermedades mendelianas y sus correspondientes genes. 

\medskip
Para simular un estudio real, en primer lugar se decide el número de familias que lo componen y el número de genes por familia. A continuación, se escoge al azar una enfermedad de OMIM que contenga un número de genes mayor o igual al de familias. Seguidamente, se le asigna a cada familia un gen de la enfermedad seleccionada. Por último, se añade a cada familia un conjunto de genes aleatorios, componiendo así el set final de genes candidatos.

\begin{equation}
 c_i = [T_i,R_1,R_2,...,R_k]
\end{equation}
\\

donde $c_i$ se corresponde con la lista de candidatos aportada por la familia $iesima$, compuesta por un gen de enfermedad $T_i$ y un conjunto de $k$ genes escogidos de forma aleatoria.

\medskip
Este conjunto de genes es el equivalente al que habría seleccionado una familia después de haber procesado los individuos que la componen.

\medskip
Con el esquema de simulación planteado, se ha confeccionado una serie de experimentos destinados a evaluar algunos aspectos críticos de la metodología de priorización. A continuación se describen las diferentes tandas de simulación.

\subsubsection{Medidas de distancia}
En primer lugar, se ha realizado una tanda de experimentos con el fin de evaluar la eficacia de las distintas medidas de distancia propuestas. Los experimentos planteados son los siguientes:
	
\bigskip
\begin{small}
\begin{center}
\begin{table}[H]
\begin{tabular}{cccc}
\textbf{distancia} & \textbf{repeticiones} & \textbf{familias} & \textbf{genes por familia}\tabularnewline 
\hline
\hline
\addlinespace[0.2cm]
\textbf{SP} & 100 & 3 & 150\tabularnewline
\textbf{ID} & 100 & 3 & 150\tabularnewline
\textbf{RW} & 100 & 3 & 150\tabularnewline
\addlinespace[0.2cm]
\hline
\end{tabular}
\caption{Tandas de simulación planteadas para evaluar las 3 medidas de distancia propuestas}
\label{tab:tabla_distancias}
\end{table}

\end{center}
\end{small}
\medskip

\subsubsection{Número de familias}
El número de familias empleadas en el estudio es un parámetro crítico que va a influir en la fiabilidad de los resultados, ya que a mayor número de familias, menor es la probabilidad de encontrar genes aleatorios en todas las familias. Concretamente, se plantean las siguientes simulaciones.

\bigskip
\begin{small}
\begin{center}
\begin{table}[H]
\begin{tabular}{cccc}
\textbf{distancia} & \textbf{repeticiones} & \textbf{familias} & \textbf{genes por familia}\tabularnewline 
\hline
\hline
\addlinespace[0.2cm]
SP	 & 100 & \textbf{3} & 100 \tabularnewline 
SP	 & 100 & \textbf{4} & 100 \tabularnewline
SP	 & 100 & \textbf{5} & 100 \tabularnewline
SP	 & 100 & \textbf{7} & 100 \tabularnewline
SP	 & 100 & \textbf{10} & 100 \tabularnewline
\addlinespace[0.2cm]
\hline
\addlinespace[0.2cm]
ID	 & 100 & \textbf{3} & 100 \tabularnewline 
ID	 & 100 & \textbf{4} & 100 \tabularnewline
ID	 & 100 & \textbf{5} & 100 \tabularnewline
ID	 & 100 & \textbf{7} & 100 \tabularnewline
ID	 & 100 & \textbf{10} & 100 \tabularnewline
\addlinespace[0.2cm]
\hline
\addlinespace[0.2cm]
RW & 100 & \textbf{3} & 100 \tabularnewline 
RW & 100 & \textbf{4} & 100 \tabularnewline
RW & 100 & \textbf{5} & 100 \tabularnewline
RW & 100 & \textbf{7} & 100 \tabularnewline
RW & 100 & \textbf{10} & 100 \tabularnewline
\addlinespace[0.2cm]
\hline
\end{tabular}
\caption{Tandas de simulación planteadas para evaluar la influencia del número de familias sobre el resultado}
\label{tab:tabla_familias}
\end{table}
\end{center}
\end{small}
\bigskip

\subsubsection{Número de genes por familia}
Otro de los parámetros importantes a evaluar lo constituye el número de genes por familia, ya que a mayor número de genes, más ruido entrará en el sistema y por tanto, más complicada será la priorización. Los experimentos planteados son los siguientes.

\bigskip
\begin{small}
\begin{center}
\begin{table}[H]
\begin{tabular}{cccc}
\textbf{distancia} & \textbf{repeticiones} & \textbf{familias} & \textbf{genes por familia}\tabularnewline 
\hline
\hline
\addlinespace[0.2cm]
SP	 & 100 & 3 & \textbf{50} \tabularnewline
SP	 & 100 & 3 & \textbf{100} \tabularnewline
SP	 & 100 & 3 & \textbf{200} \tabularnewline
SP	 & 100 & 3 & \textbf{500} \tabularnewline
\addlinespace[0.2cm]
\hline
\addlinespace[0.2cm]
ID	 & 100 & 3 & \textbf{50} \tabularnewline
ID	 & 100 & 3 & \textbf{100} \tabularnewline
ID	 & 100 & 3 & \textbf{200} \tabularnewline
ID	 & 100 & 3 & \textbf{500} \tabularnewline
\addlinespace[0.2cm]
\hline
\addlinespace[0.2cm]
RW  & 100 & 3 & \textbf{50} \tabularnewline
RW	 & 100 & 3 & \textbf{100} \tabularnewline
RW	 & 100 & 3 & \textbf{200} \tabularnewline
RW	 & 100 & 3 & \textbf{500} \tabularnewline
\addlinespace[0.2cm]
\hline
\end{tabular}
\caption{Tandas de simulación planteadas para evaluar la influencia del número de genes por familia sobre el resultado}
\label{tab:tabla_genes}
\end{table}
\end{center}
\end{small}
\bigskip

\subsubsection{Solapamiento entre familias}
En la práctica, las familias de un estudio pueden coincidir en el gen de la enfermedad. A nivel computacional, esto provoca que el término asociado a la intersección directa tenga más peso que el término de interacción. Para este fin, se ha empleado otra tanda de experimentos donde se evalúa el grado de solapamiento entre familias:

\bigskip
\begin{small}
\begin{center}
\begin{table}[H]
\begin{tabular}{ccccc}
\textbf{distancia} & \textbf{repeticiones} & \textbf{familias} & \textbf{genes por familia} & \textbf{genes de enfermedad} \tabularnewline 
\hline
\hline
\addlinespace[0.2cm]
SP	 & 100 & 5 & 100 & \textbf{5} \tabularnewline 
SP	 & 100 & 5 & 100 & \textbf{4} \tabularnewline
SP	 & 100 & 5 & 100 & \textbf{3} \tabularnewline
SP	 & 100 & 5 & 100 & \textbf{2} \tabularnewline
SP	 & 100 & 5 & 100 & \textbf{1} \tabularnewline
\addlinespace[0.2cm]
\hline
\addlinespace[0.2cm]
ID	 & 100 & 5 & 100 & \textbf{5} \tabularnewline 
ID	 & 100 & 5 & 100 & \textbf{4} \tabularnewline
ID	 & 100 & 5 & 100 & \textbf{3} \tabularnewline
ID	 & 100 & 5 & 100 & \textbf{2} \tabularnewline
ID	 & 100 & 5 & 100 & \textbf{1} \tabularnewline
\addlinespace[0.2cm]
\hline
\addlinespace[0.2cm]
RW	 & 100 & 5 & 100 & \textbf{5} \tabularnewline 
RW	 & 100 & 5 & 100 & \textbf{4} \tabularnewline
RW	 & 100 & 5 & 100 & \textbf{3} \tabularnewline
RW	 & 100 & 5 & 100 & \textbf{2} \tabularnewline
RW	 & 100 & 5 & 100 & \textbf{1} \tabularnewline
\addlinespace[0.2cm]
\hline
\end{tabular}
\caption{Tandas de simulación planteadas para evaluar las 3 medidas de distancia propuestas}
\label{tab:tabla_disease}
\end{table}

\end{center}
\end{small}
\bigskip

\subsubsection{Otras ponderaciones complementarias}
Por último, se ha considerado importante evaluar el grado de mejora ofrecido por la ponderación del estadístico de priorización con factores relativos al gen. En este caso, se han probado el estadístico de centralidad grado, y el algoritmo PageRank. La siguiente tanda de experimentos se ha diseñado para probar su influencia:


\bigskip
\begin{small}
\begin{center}
\begin{table}[H]
\begin{tabular}{ccccc}
\textbf{distancia} & \textbf{repeticiones} & \textbf{familias} & \textbf{genes por familia} & \textbf{complementario} \tabularnewline 
\hline
\hline
\addlinespace[0.2cm]
SP	 & 100 & 3 & 100 & \textbf{ninguno} \tabularnewline 
SP	 & 100 & 3 & 100 & \textbf{grado} \tabularnewline
SP	 & 100 & 3 & 100 & \textbf{pagerank} \tabularnewline
SP	 & 100 & 3 & 100 & \textbf{grado + pagerank} \tabularnewline
\addlinespace[0.2cm]
\hline
\addlinespace[0.2cm]
ID	 & 100 & 3 & 100 & \textbf{ninguno} \tabularnewline 
ID	 & 100 & 3 & 100 & \textbf{grado} \tabularnewline
ID & 100 & 3 & 100 & \textbf{pagerank} \tabularnewline
ID	 & 100 & 3 & 100 & \textbf{grado + pagerank} \tabularnewline
\addlinespace[0.2cm]
\hline
\addlinespace[0.2cm]
RW & 100 & 3 & 100 & \textbf{ninguno} \tabularnewline 
RW & 100 & 3 & 100 & \textbf{grado} \tabularnewline
RW & 100 & 3 & 100 & \textbf{pagerank} \tabularnewline
RW & 100 & 3 & 100 & \textbf{grado + pagerank} \tabularnewline
\addlinespace[0.2cm]
\hline
\end{tabular}
\caption{Tandas de simulación planteadas para evaluar las 3 medidas de distancia propuestas}
\label{tab:tabla_otras}
\end{table}
\end{center}
\end{small}
\bigskip

	\subsection{Caso de uso}
	
La validación con simulaciones ha sido complementada con un caso real. Se trata de xx individuos correspondientes a 3 familias (figura xx) cuyo con el síndrome xxxx, cuyo gen causante es conocido.



\section{Implementación}

El método propuesto ha sido implementado en el lenguaje de programación R \cite{R}. Para la gestión de redes se ha empleado el paquete iGraph. Tanto las simulaciones, como el procesamiento de las secuencias del caso de uso han sido ejecutados en un cluster formado por 4 máquinas de 48 Gb y 8 procesadores. 

\medskip
La redacción del presente trabajo ha sido confeccionada bajo el lenguaje Latex \cite{latex-project,latex-book}, por medio del editor Texmaker \cite{texmaker}, bajo una máquina con sistema operativo Mac OSX.


